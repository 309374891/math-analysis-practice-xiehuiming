\documentclass[12pt]{ctexart}
\usepackage{geometry}
 \geometry{
 a4paper,
 total={170mm,257mm},
 left=20mm,
 top=20mm,
 }

\begin{document}
\large
1.3.2 练习题
1.关于bernoulli不等式的推广:

(1).当$-2\le{h}\le-1$时,bernoulli不等式$(1+h)^n\ge{1+nh}$仍然成立。

(2).当$h\ge0$时,成立不等式$(1+h)^n\ge{\frac{n(n-1)h^2}{2}}$,并推广之。

(3)证明:当$a_i>1-(i=0,1,2,...,n)$且同号,则成立不等式

$\prod\limits_{i=1}^n(1+a_i)\ge{1+\sum\limits_{i=1}^na_i}$
\\\\
解:\\
(1).证明:
当$-2\le{h}\le-1$时,$-1\le{1+h}\le0$,所以有$-1\le{(1+h)}^n\le0$

$(1+h)^n-1=h(1+(1+h)+{(1+h)}^2+...+{(1+h)^n-1})\ge{1+nh}$
\\\\
(2).证明:

当n=1,2时容易验证不等式成立。当n大于2的时候:

$(1+h)^n=1+C_n^1h+C_n^2h^2+...+h^n=1+nh+\frac{n(n-1)}{2}h^2+...+h^n>\frac{n(n-1)h^2}{2}$
\\\\
(3).证明:

用数学归纳法,当n=1时容易验证不等式成立,假设当n=k时不等式成立,即:

$\prod\limits_{i=1}^{k}(1+a_i)\ge{1+\sum\limits_{i=1}^ka_i}$

记$A_k=\prod\limits_{i=1}^{k}(1+a_i),B_k=\sum\limits_{i=1}^ka_i$

当n=k+1时,$A_{k+1}=A_k\times{(1+a_{k+1})}\ge{(1+B_k)}(1+a_{k+1})$

$=1+B_k+a_{k+1}+B_ka_{k+1}1+b_{k+1}+B_ka_{k+1}\ge{1+B_{k+1}}$

不等式成立,得证。

2.阶乘$n!$在数学分析以及其他课程中经常出现,以下是几个有关的不等式,它们都可以从平均值不等式得到。\\\\
(1).当$n>1$时,$n!<(\frac{n+1}{2})^n$;

(2).利用$(n!)^2=(n\cdot 1)((n-1)\cdot 2)...(1\cdot n)$证明当n>1时成立:

$n!<(\frac{n+2}{\sqrt{6}})^n$;

(3).比较(1)和(2)中的两个不等式优劣,并说明原因;

(4).对任意实数r成立$(\sum\limits_{k=1}^nk^r)^n\ge{n^n(n!)^r}$

解:\\
(1).证:

$n!<(\frac{1+2+...+n}{n})^n=(\frac{\frac{n(n+1)}{2}}{n})^n=(\frac{n+1}{2})^n$,得证。

(2).证:


由$k\cdot{(n-k+1)}<(\frac{k+n-k+1}{2})^2=(\frac{n+1}{2})^2$得:

由$k\cdot{(n-k+1)}=nk-k^2+k$得:

$(n!)^2=(1\cdot n)(2\cdot (n-1))...(n\cdot1)<(\frac{\sum\limits_{k=1}^nnk-\sum\limits_{k=1}^{n}k^2+\sum\limits_{k=1}^nk}{n})^n$

$=(\frac{n\cdot{\frac{n(n+1)}{2}}-\frac{n(n+1)(2n+1)}{6}+\frac{n(n+1)}{2}}{n})^n=(\frac{(n+1)}{6}\cdot{(3n-(2n+1)+3)})^n$

$=(\frac{(n+1)}{6}\cdot{(n+2)})^n<(\frac{(n+2)^2}{6})^n$

两边开方得:$n!<(\frac{n+2}{\sqrt{6}})^n$

(3).(2)中的不等式比(1)中的不等式更优,因为$\frac{n+2}{\sqrt{6}}<\frac{n+1}{2}$

(4).证:

$(\sum\limits_{k=1}^nk^r)^n\ge{(n\sqrt[n]{\prod\limits_{k=1}^nk^r})^n}=n^n(n!)^r$

3.证明几何平均值-调和平均值不等式:$(\prod\limits_{k=1}^n)^{\frac{1}{n}}\ge{\frac{n}{\sum\limits_{k=1}^n{\frac{1}{a_k}}}}$

证:

因为:$\sum\limits_{k=1}^n{\frac{1}{a_k}}\ge{n\sqrt[n]{\frac{1}{a_1}\frac{1}{a_2}...\frac{1}{a_n}}}=\frac{n}{({\prod\limits_{k=1}{n}})^{\frac{1}{n}}}$.所以:

$(\prod\limits_{k=1}^n)^{\frac{1}{n}}\ge{\frac{n}{\sum\limits_{k=1}^n{\frac{1}{a_k}}}}$

4.证明:当a,b,c为非负数时成立$\sqrt[3]{abc}\le{\sqrt{\frac{ab+bc+ca}{3}}}\le\frac{a+b+c}{3}$

证:

先证前半部分:${\sqrt{\frac{ab+bc+ca}{3}}}\ge{{\sqrt{\frac{3\sqrt[3]{ab\cdot{bc}\cdot{ca}}}{3}}}}=\sqrt[3]{abc}$

再证后半部分:

由$a^2+b^2\ge{2ab},b^2+c^2\ge{2bc},c^2+a^2\ge{2ca}$两边叠加得$a^2+b^2+c^2\ge{ab+bc+ca}$

两边同时加上$2ab+2bc+2ca$得$a^2+b^2+c^2 +2(ab+bc+ca)\ge{3(ab+bc+ca)}$

即$(a+b+c)^2\ge{3(ab+bc+ca)}$,开方整理即得。

5.证明以下几个不等式:

(1).$|a-b|\ge|a|-|b|$和$|a-b|\ge||a|-|b||$

(2).$|a_1|-\sum\limits_{k=2}^n|a_k|\le{|\sum\limits_{k=1}^na_k|}\le{\sum\limits_{k=1}^n|a_k|}$,又问:左边可否为$||a_1|-\sum\limits_{k=2}^n|a_k||$

(3).$\frac{|a+b|}{1+|a+b|}\le{\frac{|a|}{1+|a+b|}+\frac{|b|}{1+|a+b|}}$

(4).$|(a+b)^n-a^n|\le{(|a|+|b|)^n-|a|^n}$

解:

(1).$|a-b|+|b|\ge|a-b+b|=|a|$即$|a-b|\ge|a|-|b|$

根据前面的不等式有$|a-b|\ge|a|-|b|$,又$|a-b|=|b-a|\ge{|b|-|a|}$,即:


$-|a-b|\le||a|-|b||\le{|a-b|}$所以有$|a-b|\ge||a|-|b||$

(2).先证后半部分:

用数学归纳法:

当n=1,2时不等式明显成立。假设当n=k时不等式成立,即:

$|\sum\limits_{i=1}^{k}a_i|\le\sum\limits_{i=1}^{k}|a_i|$

则当n=k+1时有:

$\sum\limits_{i=1}^{k+1}|a_i|=\sum\limits_{i=1}^{k}|a_i|+|a_{k+1}|\ge{|\sum\limits_{i=1}^{k}a_i|+|a_{k+1}|}\ge{|\sum\limits_{i=1}^{k}a_i+a_{k+1}|}=|\sum\limits_{i=1}^{k+1}a_i|$

不等式对n=k+1成立。

再证前半部分:

根据前面所证,有:

$|\sum\limits_{k=1}^{n}a_k|+\sum\limits_{k=2}^{n}|a_k|=|\sum\limits_{k=1}^{n}a_k|+\sum\limits_{k=2}^{n}|-a_k|\ge{|\sum\limits_{k=1}^{n}a_k+\sum\limits_{k=2}^{n}(-a_k)|}=|a_1|$即:

$|a_1|-\sum\limits_{k=2}^{n}|a_k|\le{|\sum\limits_{k=1}^{n}a_k|}$.左边不能为$||a_1|-\sum\limits_{k=2}^n|a_k||$,比如取$|a_1=0,a_2=2,a_3=-9|$,不等式不成立。

(3).

当$|a+b|=0$时不等式显然成立。当$|a+b|>0$时:

$\frac{|a+b|}{1+|a+b|}=\frac{1}{\frac{1}{|a+b|}+1}\le\frac{1}{\frac{1}{|a|+|b|}+1}=\frac{|a|+|b|}{1+|a|+|b|}=\frac{|a|}{1+|a|+|b|}+\frac{|b|}{1+|a|+|b|}\le\frac{|a|}{1+|a|}+\frac{|b|}{1+|b|}$

(4).$|(a+b)^n-a^n|=|\sum\limits_{k=0}^nC_n^ka^{n-k}b^k-a^n|=|\sum\limits_{k=1}^nC_n^ka^{n-k}b^k|\le\sum\limits_{k=1}^nC_n^k|a|^{n-k}|b|^k$

$=\sum\limits_{k=0}^nC_n^k|a|^{n-k}|b|^k-|a|^n=(|a|+|b|)^n-|a|^n$

6.试按下列提示,给出cauchy不等式的几个不通证明:

(1)用数学归纳法。

(2)用lagrange恒等式:$\sum\limits_{k=1}^{n}a_k^2\cdot\sum\limits_{k=1}^{n}b_k^2-(\sum\limits_{k=1}^{n}|a_kb_k|)^2=\frac{1}{2}\sum\limits_{k=1}^n\sum\limits_{i=1}^n(|a_k||b_i|-|a_i||b_k|)^2$。

(3).利用不等式$|AB|\le\frac{A^2+B^2}{2}$

(4).构造复的辅助数列:$c_k=a_k^2-b_k^2+2i|a_kb_k|,k=1,2,...n$,再利用

$|\sum\limits_{k=1}^nc_k|\le\sum\limits_{k=1}^n|c_k|$\\\\
解:

(1).用数学归纳法:当n=2时容易验证不等式成立。当n>2时,假设当n=k时不等式成立,即:

$\sum\limits_{i=1}^ka_i^2\sum\limits_{i=1}^kb_i^2\ge(\sum\limits_{i=1}^{k}a_ib_i)^2$,当n=k+1时:

$\sum\limits_{i=1}^{k+1}a_i^2\sum\limits_{i=1}^{k+1}b_i^2=(\sum\limits_{i=1}^ka_i^2+a_{k+1}^2)(\sum\limits_{i=1}^{k}b_i^2+b_{k+1}^2)=\sum\limits_{i=1}^ka_i^2\sum\limits_{i=1}^kb_i^2+a_{k+1}^2\sum\limits_{i=1}^kb_i^2+b_{k+1}^2\sum\limits_{i=1}^ka_i^2+a_{k+1}^2b_{k+1}^2$

$\ge(\sum\limits_{i=1}^{k}a_ib_i)^2+2\sqrt{a_{k+1}^2b_{k+1}^2\sum\limits_{i=1}^ka_i^2\sum\limits_{i=1}^kb_i^2}+a_{k+1}^2b_{k+1}^2$

$\ge(\sum\limits_{i=1}^{k}a_ib_i)^2+2a_{k+1}b_{k+1}\sum\limits_{i=1}^ka_ib_i+a_{k+1}^2b_{k+1}^2$

$=(\sum\limits_{i=1}^{k}a_ib_i+a_{k+1}b_{k+1})^2=(\sum\limits_{i=1}^{k+1}a_ib_i)^2$

(2).等式右边非负,整理即得。

(3).如果$\sum\limits_{k=1}^na_k^2=0$,则不等式显然成立,当$\sum\limits_{k=1}^na_k^2!=0$时,记$\lambda^2=\sqrt{\frac{\sum\limits_{k=1}^{n}b_k^2}{\sum\limits_{k=1}^{n}a_i^2}}$

$\lambda^2a_k^2+\frac{b_k^2}{\lambda^2}\ge2a_kb_k$所以$\lambda^2\sum\limits_{k=1}^na_k^2+\frac{1}{\lambda^2}\sum\limits_{k=1}^nb_k^2\ge2\sum\limits_{k=1}^na_kb_k$

即:$\sqrt{\sum\limits_{k=1}^na_k^2}\sqrt{\sum\limits_{k=1}^nb_k^2}\ge\sum\limits_{k=1}^na_kb_k$

(4).$|\sum\limits_{k=1}^nc_k|=|\sum\limits_{k=1}^na_k^2-\sum\limits_{k=1}^nb_k^2+2i\sum\limits_{k=1}^n|a_kb_k||=\sqrt{(\sum\limits_{k=1}^na_k^2-\sum\limits_{k=1}^nb_k^2)^2+4(\sum\limits_{k=1}^n|a_kb_k|)^2}$

$\sum\limits_{k=1}^n|a_k|=\sum\limits_{k=1}^n(\sqrt{(a_k^2-b_k^2)^2+4(|a_kb_k|^2)}=\sum\limits_{k=1}^{n}\sqrt{(a_k^2+b_k^2)^2}=\sum\limits_{k=1}^{n}a_k^2+\sum\limits_{k=1}^{n}b_k^2$

由$|\sum\limits_{k=1}^{n}c_k|\le\sum\limits_{k=1}^n|c_k|$,即:$\sqrt{(\sum\limits_{k=1}^na_k^2-\sum\limits_{k=1}^nb_k^2)^2+4(\sum\limits_{k=1}^n|a_kb_k|)^2}\le\sum\limits_{k=1}^{n}a_k^2+\sum\limits_{k=1}^{n}b_k^2$

两边平方得:$(\sum\limits_{k=1}^na_k^2-\sum\limits_{k=1}^nb_k^2)^2+4(\sum\limits_{k=1}^n|a_kb_k|)^2\le(\sum\limits_{k=1}^{n}a_k^2)^2+(\sum\limits_{k=1}^{n}b_k^2)^2+2\sum\limits_{k=1}^{n}a_k^2\sum\limits_{k=1}^{n}b_k^2$

即:$(\sum\limits_{k=1}^na_k^2)^2+(\sum\limits_{k=1}^nb_k^2)^2-2\sum\limits_{k=1}^na_k^2\sum\limits_{k=1}^nb_k^2+4(\sum\limits_{k=1}^n|a_kb_k|)^2\le(\sum\limits_{k=1}^{n}a_k^2)^2+(\sum\limits_{k=1}^{n}b_k^2)^2+2\sum\limits_{k=1}^{n}a_k^2\sum\limits_{k=1}^{n}b_k^2$

整理即得。

7.用向前-向后数学归纳法证明:$0<x_i\le\frac{1}{2},i=1,2,...,n,$则

$\frac{\prod\limits_{i=1}^nx_i}{(\sum\limits_{i=1}^nx_i)^n}\le\frac{\prod\limits_{i=1}^n(1-x_i)}{[\sum\limits_{i=1}^n(1-x_i)]^n}$

证:

原不等式等价于:$\frac{\prod\limits_{i=1}^{n}x_i}{\prod\limits_{i=1}^{n}(1-x_i)}\le\frac{(\sum\limits_{i=1}^{n}x_i)^n}{[\sum\limits_{i=1}^{n}(1-x_i)]^n}$


当n=1时,不等式明显成立。当n=2时,不等式等价于:

$\iff{x_1x_2(1-x_1+1-x_2)^2}\le{(x_1+x_2)^2(1-x_1)(1-x_2)}$

$\iff{x_1x_2(4+x_1^2+x_2^2-4x_1-4x_2+2x_1x_2)}\le{(x_1^2+2x_1x_2+x_2^2)(1-x_1-x_2+x_1x_2)}$

$\iff4x_1x_2+x_1^3x_2+x_1x_2^3-4x_1^2x_2-4x_1x_2^2+2x_1^2x_2^2$

$\le{}x_1^2-x_1^3-x_1^2x_2+x_1^3x_2+2x_1x_2-2x_1^2x_2-2x_1x_2^2+2x_1^2x_2^2+x_2^2-x_1x_2^2-x_2^3+x_1x_2^3$

$=x_1^2+x_2^2-x_1^3-x_2^3-3x_1^2x_2-3x_1x_2^2+x_1^3x_2+x_1x_2^3+2x_1x_2+2x_1^2x_2^2$

$\iff2x_1x_2-x_1^2x_2-x_1x_2^2\le{}x_1^2+x_2^2-x_1^3-x_2^3$

$\iff{}x_1^2+x_2^2-x_1^3-x_2^3-2x_1x_2+x_1^2x_2+x_1x_2^2\ge0$

$\iff(x_1-x_2)^2-(x_1+x_2)(x_1^2-x_1x_2+x_2^2)+x_1x_2(x_1+x_2)\ge0$

$\iff(x_1-x_2)^2-(x_1+x_2)(x_1-x_2)^2\ge0$

$\iff(x_1-x_2)^2(1-x_1-x_2)\ge0$

由于$x_i\le\frac{1}{2}$所以$1-x_1-x_2\ge0$,所以当n=2时不等式成立:

$\frac{x_1x_2}{(1-x_1)(1-x_2)}\le\frac{(x_1+x_2)^2}{[(1+x_1)+(1+x_2)]^2}--------(1)$

假设当$n=2^m$时不等式成立,即:

$\frac{\prod\limits_{i=1}^{2^{m}}x_i}{\prod\limits_{i=1}^{2^{m}}(1-x_i)}\le\frac{(\sum\limits_{i=1}^{2^{m}}x_i)^{2^{m}}}{[\sum\limits_{i=1}^{2^{m}}(1-x_i)]^{2^{m}}}$

又对任意的n有$\frac{\sum\limits_{i=1}^{n}x_i}{\sum\limits_{i=1}^{n}(1-x_i)}=\frac{\frac{\sum\limits_{i=1}^{n}{x_i}}{n}}{1-\frac{\sum\limits_{i=1}^{n}x_i}{n}}$

则当$n=2^{m+1}$时,记$A_m=\frac{\sum\limits_{i=1}^{2^m}{x_i}}{2^m},B_m=\frac{\sum\limits_{i=2^m+1}^{2^{m+1}}{x_i}}{2^m}$:

$\frac{\prod\limits_{i=1}^{2^{m+1}}x_i}{\prod\limits_{i=1}^{2^{m+1}}(1-x_i)}$
$=\frac{\prod\limits_{i=1}^{2^{m}}x_i}{\prod\limits_{i=1}^{2^{m}}(1-x_i)\cdot}\frac{\prod\limits_{i=2^m+1}^{2^{m+1}}x_i}{\prod\limits_{i=2^m+1}^{2^{m+1}}(1-x_i)}$
$\le\frac{(\sum\limits_{i=1}^{2^{m}}x_i)^{2^{m}}}{[\sum\limits_{i=1}^{2^{m}}(1-x_i)]^{2^{m}}}\cdot\frac{(\sum\limits_{i=2^m+1}^{2^{m+1}}x_i)^{2^{m}}}{[\sum\limits_{i=2^m+1}^{2^{m+1}}(1-x_i)]^{2^{m}}}$

$=(\frac{A_m}{1-A_m}\cdot\frac{B_m}{1-B_m})^{2^m}=(\frac{A_mB_m}{(1-A_m)(1-B_m)})^{2^m}$

$\le\{[\frac{(1-A_m)(1-B_m)}{1-A_m+1-B_m}]^2\}^{2^m}=[\frac{A_m+B_m}{2-A_m-B_m}]^{2^{m+1}}$(根据(1)式)

$=\{\frac{\sum\limits_{i=1}^{2^m}x_i+\sum\limits_{i=2^m+1}^{2^{m+1}}x_i} {2^{m+1}-(\sum\limits_{i=1}^{2^m}x_i+\sum\limits_{i=2^m+1}^{2^{m+1}}x_i)}\}^{2^{m+1}}$

$=(\frac{\sum\limits_{i=1}^{2^{m+1}}x_i}{\sum\limits_{i=1}^{2^{m+1}}(1-x_i)})^{2^{m+1}}$

即不等式当$n=2^{m+1}$时成立。

证向后部分:

假设当n=k时不等式成立:
$\frac{\prod\limits_{i=1}^{k}x_i}{\prod\limits_{i=1}^{k}(1-x_i)}\le\frac{(\sum\limits_{i=1}^{k}x_i)^k}{[\sum\limits_{i=1}^{k}(1-x_i)]^k}$

记$A=\frac{\sum\limits_{i=1}^{k-1}{x_i}}{k-1}$则$\frac{A}{1-A}=\frac{\sum\limits_{i=1}^{k-1}{x_i}}{\sum\limits_{i=1}^{k-1}{(1-x_i)}}$

当n=k-1时:

$\frac{\prod\limits_{i=1}^{k-1}{x_i}}{\prod\limits_{i=1}^{k-1}(1-x_i)}\cdot\frac{A}{1-A}\le\frac{(\sum\limits_{i=1}^{k-1}{x_i}A)^k}{[\sum\limits_{i=1}^{k-1}{(1-x_i)}+(1-A)]^k}$

$=\frac{(k\sum\limits_{i=1}^{k-1}x_i)^k}{[{(k-1)}^2-(k-1)\sum\limits_{i=1}^{k-1}{x_i}+(k-1)-\sum\limits_{i=1}^{k-1}{x_i}]^k}$

$=\frac{(k\sum\limits_{i=1}^{k-1}x_i)^k}{[{k(k-1)}-k\sum\limits_{i=1}^{k-1}{x_i}]^k}$

$=\frac{(\sum\limits_{i=1}^{k-1}x_i)^k}{[{(k-1)}-\sum\limits_{i=1}^{k-1}{x_i}]^k}$

$=\frac{(\sum\limits_{i=1}^{k-1}x_i)^k}{[\sum\limits_{i=1}^{k-1}{(1-x_i)}]^k}$

所以:

$\frac{\prod\limits_{i=1}^{k-1}{x_i}}{\prod\limits_{i=1}^{k-1}(1-x_i)}\le\frac{(\sum\limits_{i=1}^{k-1}x_i)^k}{[\sum\limits_{i=1}^{k-1}{(1-x_i)}]^k}\cdot\frac{1-A}{A}$
$=\frac{(\sum\limits_{i=1}^{k-1}x_i)^{k-1}}{[\sum\limits_{i=1}^{k-1}{(1-x_i)}]^{k-1}}$

即不等式对于n=k-1成立,据以上所知,不等式对一切n都成立。

8.设a,c,g,t均为非负数,a+c+g+t=1,证明:$a^2+c^2+g^2+t^2\ge\frac{1}{4}$,

且其中的等号成立的充分必要条件是$a=c=g=t=\frac{1}{4}$.

证:

由均值不等式得:

$\sqrt{\frac{a^2+c^2+g^2+t^2}{4}}\ge\frac{a+c+g+t}{4}=\frac{1}{4}$,整理即得。
\end{document}