\documentclass[12pt]{ctexart}
\usepackage{geometry}
 \geometry{
 a4paper,
 total={170mm,257mm},
 left=20mm,
 top=20mm,
 }

\begin{document}
\noindent2.1.2 思考题

1.数列收敛有很多等价的定义,比如:

(1).数列${a_n}$收敛于a$\iff\forall{\epsilon}>0, \exists{N}\in{N_+},\forall{n}\ge{N,}$,成立$|a_n-a|<\epsilon$;

(2).数列${a_n}$收敛于a$\iff\forall{m}\in{N_+},\exists{N}\in{N_+},\forall{n}>{N,}$成立$|a_n-a|<\frac{1}{m}$;

(3).数列${a_n}$收敛于a$\iff\forall\epsilon>0,\exists{N}\in{N_+},\forall{n}>N,$成立$|a_n-a|<K\epsilon$, K是一个与$\epsilon$和n无关的常数;

解:
(1).先证充分性:

根据数列收敛的定义有:

$\forall\epsilon>0,\exists{N_1}\in{N_+},\forall{n}>{N_1},|a_n-a|<\epsilon$,取$N=N_1+1$,

则$\forall{\epsilon}>0, \exists{N}\in{N_+},\forall{n}\ge{N},|a_n-a|<\epsilon$

再证必要性:

由于$\forall{\epsilon}>0, \exists{N}\in{N_+},\forall{n}\ge{N},|a_n-a|<\epsilon$对$n\ge{N}$成立,

所以明显对于$n>N$的情况也成立,即:

$\forall{\epsilon}>0, \exists{N}\in{N_+},\forall{n}>{N},|a_n-a|<\epsilon$

(2).先证充分性:

对任意的$m\in{N_+}$,取$\epsilon=\frac{1}{m}$易得。

再证必要性:

对任意的$\epsilon>0$,总存在$m\in{N_+}$,使得$\epsilon\ge\frac{1}{m}$,所以$|a_n-a|<\frac{1}{m}\le\epsilon$

(3).先证充分性:

根据数列收敛有:$\forall\epsilon_1>0,\exists{N}\in{N_+},\forall{n}>N,|a_n-a|<\epsilon_1$

对任意的$\epsilon>0$,总存在$\epsilon_1=K\epsilon$.带入上式得证。

再证必要性:

$\forall\epsilon_1>0,\exists{N}\in{N_+},\forall{n}>N,|a_n-a|<K\epsilon_1$

对任意的$\epsilon>0$,总存在$\epsilon_1=\frac{\epsilon}{K}$.带入上式得证。

2.问:在数列收敛的定义中,N是否$\epsilon$的函数?

答:否。N和$\epsilon$可能有关,但不是函数关系

比如当$n>N(\epsilon),|a_n-a|<\epsilon$,此时N可以取$N(\epsilon)+1,N(\epsilon)+2,...$,不一定非要取$N(\epsilon)$。

3.判断正确与否:若${a_n}$收敛,则有$\lim\limits_{n\to\infty}(a_{n+1}-a_{n})=0$和$\lim\limits_{n\to\infty}\frac{a_{n+1}}{a_n}=1$.

假设$\lim\limits_{n\to\infty}a_n=a$.

$\forall\epsilon>0,\exists{N}\in{N_+},\forall{n}>N,|a_n-a|<\epsilon,|a_{n+1}-a|<\epsilon$

则$|a_{n+1}-a_{n}|=|a_{n+1}-a-a_{n}+a|\le|a_{n+1}+a|+|a_{n}-a|<2\epsilon$

根据上题的第三小题知$\lim\limits_{n\to\infty}(a_{n+1}-a_{n})=0$成立。

若$a_n$有可能为0,则$\frac{a_{n+1}}{a_n}$没有意义。

当$a_n!=0$时,不成立,比如$a_{2n}=\frac{1}{n}, a_{2n+1}=\frac{1}{n^2}$.

4.设收敛数列的每一项都是整数,问:该数列有什么特殊性质?

答:根据上一题结论,$\lim\limits_{n\to\infty}(a_{n+1}-a_n)=0$取$\epsilon=\frac{1}{2}$,知$\exists{N}\in{N_+},|a_{n+1}-a_n|<\frac{1}{2}$

又$a_{n+1},a_n$均为整数,所以$|a_{n+1}-a_n|$也是整数,所以$|a_{n+1}-a_n|$只能为0,所以该数列从某项后只能是常数数列。

5.问:收敛数列是否一定是单调数列?无穷小量是否一定是单调数列?

答:两个问题答案均为否。比如取$a_n=(-1)^n\frac{1}{n}$,数列收敛于0,但是不单调。


6.问:一个很小很小的量,例如取一米为单位长度时几个纳米大小的量,是否为无穷小量?如何刻画一个无穷小量的大小?

答:一个很小很小的量不是无穷小量。无穷小量是一个极限为0的数列,不是有限数的量。用无穷小量的阶可以刻画无穷小量的大小。

7.问:正无穷大数列是否一定单调增加?无界数列是否一定是无穷大量?

答:正无穷大数列不一定单调增加,比如:

$
a_n = \left\{ \begin{array}{ll}
n & \textrm{$n=2k$}\\
\frac{n}{2} & \textrm{n=2k+1}
\end{array} \right.
$

无界数列不一定是无穷大量。
比如:

$a_n = \left\{ \begin{array}{ll}n & \textrm{$n=2k$}\\0 & \textrm{n=2k+1}\end{array} \right.$

8.判断正确与否:非负数列的极限是非负数,正数的极限是正数。

答:非负数列的极限一定是非负数。

反证法:假设$\lim\limits_{n\to\infty}a_n=a<0$,由于$a_n\ge$0|所以$|a_n-a|=a_n+|a|\ge|a|$

根据极限定义取$\epsilon=|a|$则不存在$N\in{N_+},\forall{n}>N,|a_n-a|<\epsilon$

矛盾,所以$a\ge0$

正数的极限不一定是正数,比如:

$a_n=\frac{1}{n}$的极限为0.

2.1.5 练习题

1.按极限定义证明:

(1).$\lim\limits_{n\to\infty}\frac{3n^2}{n^2-4}=3$

(2).$\lim\limits_{n\to\infty}\frac{\sin{n}}{n}=1$

(3).$\lim\limits_{n\to\infty}{(1+n)^{\frac{1}{n}}}=1$

(4).$\lim\limits_{n\to\infty}\frac{a^n}{n!}=0(a>0)$

解:
(1).$|\frac{3n^2}{n^2-4}-3|=|\frac{3n^2-(3n^2-12)}{n^2-4}|=|\frac{12}{(n-2)(n+2)}|$

当n>2时,(n-2)(n+2)>n,所以$|\frac{12}{(n-2)(n+2)}|<\frac{12}{n}$,所以对于给定$\epsilon$取$\epsilon>\frac{12}{n}\textrm{得}n>\frac{12}{\epsilon}$

取$N=max\{2,[\frac{12}{\epsilon}]+1\}$即可。

(2).$|\frac{\sin{n}}{n}-1|=|\frac{\sin{n}-1}{n}|\le|\frac{2}{n}|\textrm{(因为}-2\le{1-\sin{n}}\le0)$

对于给定的$\epsilon>0$取$N=[\frac{2}{\epsilon}]+1$即可。

(3).令$y_n={(1+n)^{\frac{1}{n}}}-1>-1$,由$(n-1)\ge\frac{n+1}{2}$

得$1+n={(1+y_n)}^n\ge\frac{n(n-1)}{2}y_n^2\ge\frac{n(n+1)}{4}y_n^2$

得$y_n\le\sqrt{\frac{4}{n}}$

对于给定的$\epsilon>0$,取$N=[\sqrt{\frac{4}{\epsilon}}]+1$即可。

(4).当$a\ge1$1时,$|\frac{a^n}{n!}|\le\frac{1}{n!}\le\frac{1}{n}$,对于给定的$\epsilon$,取$N=[\frac{1}{\epsilon}]+1$即可。

当$a\le1$时,$|\frac{a^n}{n!}|\le|\frac{1}{n}|$,所以对给定的$\epsilon$,取$N=[\frac{1}{\epsilon}]+1$即可。



当a>1时

当$n>N_1^{N_1}+1$时,记$n=N_1^{N_1}+k(k=1,2,...),\textrm{则}a^n=a^{N_1}a^k$

$n!>n(N_1^{N_1}+1)(N_1^{N_1}+2)...(N_1^{N_1}+k)>nN_1^{N_1+k}>na^{N_1}a^k=na^n$

$|\frac{a^n}{n!}|<|\frac{a^n}{na^n}|=\frac{1}{n}$

所以对于给定的$\epsilon>0$取$N=max\{N_1^{N_1}+1,[\frac{1}{\epsilon}+1]\}$即可。

2.设$a_n\ge0,n\in{N_+}$,数列$a_n$收敛于a,则$\lim\limits_{n\to\infty}\sqrt{a_n}=\sqrt{a}$.

证:对任意的$\epsilon>0,\exists{N},\forall{}n>N,|a_n-a|<\epsilon$

由于$a_n$极限存在,所以$a_n$必有界。记为:$|a_n|\le{M}$,所以$\sqrt{a_n}\le\sqrt{M}$

由$|a_n-a|<\epsilon$,即$|(\sqrt{a_n}-\sqrt{a})(\sqrt{a_n}+\sqrt{a})|<\epsilon$

所以$|\sqrt{a_n}-\sqrt{a}|<\frac{\epsilon}{\sqrt{a_n}+\sqrt{a}}<\frac{\epsilon}{\sqrt{M}+\sqrt{a}}=\frac{1}{\sqrt{M}+\sqrt{a}}\epsilon$

${\sqrt{M}+\sqrt{a}}$为常数,所以$\lim\limits_{n\to\infty}\sqrt{a_n}=\sqrt{a}$

3.若$\lim\limits_{n\to\infty}a_n=a$,则$\lim\limits_{n\to\infty}|a_n|=|a|$.反之如何?

解:因为$\forall\epsilon>0,\exists{}N,\forall{}n>N,|a_n-a|<\epsilon$

则$|a_n|-|a|\le|a_n-a|<\epsilon$,所以$\lim\limits_{n\to\infty}|a_n|=|a|$。

反之不成立,比如:$a_n={(-1)}^n$

4.下面一组题在本章中的许多极限计算中有用(并与第五章中的连续性概念有关):

(1).设P(x)是x的多项式。若$\lim\limits_{n\to\infty}a_n=a$,则$\lim\limits_{n\to\infty}P(a_n)=P(a)$;

(2).设b>0,$\lim\limits_{n\to\infty}a_n=a$,则$\lim\limits_{n\to\infty}b^{a_n}=b^a$;

(3).设b>0,$\{a_n\}$为正数列,$\lim\limits_{n\to\infty}a_n=a,a>0$,则$\lim\limits_{n\to\infty}\log_{b}a_n=\log_{b}a$;

(4).设b为实数,$\{a_n\}$为正数列,$\lim\limits_{n\to\infty}a_n=a,a>0$,则$\lim\limits_{n\to\infty}a_n^b=a^b$;

(5).设$\lim\limits_{n\to\infty}a_n=a$,则$\lim\limits_{n\to\infty}\sin{a_n}=\sin{a}$;

解:

(1).$\lim\limits_{n\to\infty}a_n=a\textrm{即:}\forall\epsilon>0,\exists{}N\in{N_+},\forall{}n>N,|a_n-a|<\epsilon$

且$a_n$有界,$|a_n|\le{M}$

记$P(x)=b_kx^k+{b_{k-1}x^{k-1}}+...+{b_{1}x}+b_0$

$|b_ia_n^i-b_ia^i|=|b_i(a_n-a)(a_n^{i-1}+a_n^{i-2}a+...+a_na^{i-2}+a^{i-1})|$

$\le|b_i||a_n-a|(|a_n^{i-1}|+|a_n^{i-2}a|+...+|a_na^{i-2}|+|a^{i-1}|)$

$\le|a_n-a||b_i|(|M^{i-1}|+|M^{i-2}a|+...+|M^{i-2}|+|a^{i-1}|)$

记$c_i=|b_i|(|M^{i-1}|+|M^{i-2}a|+...+|M^{i-2}|+|a^{i-1}|$,则:

$|P(a_n)-P(a)|=|b_k(a_n^k-a^k)+b_{k-1}(a_n^{k-1}-a^{k-1})+...b_1(a_n-a)|$

$\le|a_n-a|(c_k+c_{k-1}+...+c_1)<(c_k+c_{k-1}+...+c_1)\epsilon$

由于$(c_k+c_{k-1}+...+c_1)$是常数,所以根据定义有$\lim\limits_{n\to\infty}P(a_n)=P(a)$

(2).

当b=1时显然成立。

当b>1时:

对于任意的$0<\epsilon<b^a$,则:

$(\frac{\epsilon}{b^a}+1)>1,\log_b{(\frac{\epsilon}{b^a}+1)}>0, (1-\frac{\epsilon}{b^a})<1,\log_b{(1-\frac{\epsilon}{b^a})}<0$

则一定存在$N_1$,当$n>N_1$时,$\log_b{(1-\frac{\epsilon}{b^a})}<(a_n-a)<\log_b{(\frac{\epsilon}{b^a}+1)}$,所以有:

$1-\frac{\epsilon}{b^a}<b^{a_n-a}<b^{\log_b{(1+\frac{\epsilon}{b^a})}}=\frac{\epsilon}{b^a}+1$,即:$|b^{a_n-a}-1|<\frac{\epsilon}{b^a}$

所以有:$|b^{a_n}-b^a|=|b^a(b^{a_n-a}-1)|<b^a|\frac{\epsilon}{b^a}|=\epsilon$

即对于b>1成立$\lim\limits_{n\to\infty}b^{a_n}=b^a$

当b<1时:

对于任意的$0<\epsilon<b^a$,则:

$(\frac{\epsilon}{b^a}+1)>1,\log_b{(\frac{\epsilon}{b^a}+1)}<0, (1-\frac{\epsilon}{b^a})<1,\log_b{(1-\frac{\epsilon}{b^a})}>0$

则一定存在$N_1$,当$n>N_1$时,$\log_b{(1+\frac{\epsilon}{b^a})}<(a_n-a)<\log_b{(1-\frac{\epsilon}{b^a})}$,所以有:

$1+\frac{\epsilon}{b^a}>b^{a_n-a}>b^{\log_b{(1-\frac{\epsilon}{b^a})}}=1-\frac{\epsilon}{b^a}$,即:$|b^{a_n-a}-1|<\frac{\epsilon}{b^a}$

所以有:$|b^{a_n}-b^a|=|b^a(b^{a_n-a}-1)|<b^a|\frac{\epsilon}{b^a}|=\epsilon$

即对于b<1成立$\lim\limits_{n\to\infty}b^{a_n}=b^a$

(3).

对任意的$\epsilon>0$

当b>1时,存在$N_1$,当$n>N_1$时,$b^{\epsilon}-1>0,|a_n-a|<a(b^{\epsilon}-1)$,得$a_n<ab^{\epsilon}$

$|\log_b{a_n}-\log_b{a}|=|\log_b{\frac{a_n}{a}}|=\epsilon$

当b<1时,存在$N_2$,当$n>N_2$时,$1-b^{\epsilon}>0,|a_n-a|<a(1-b^{\epsilon})$,得$a_n>ab^{\epsilon}$

$|\log_b{a_n}-\log_b{a}|=|\log_b{\frac{a_n}{a}}|=\epsilon$

(4).

对任意的$\epsilon>0,(a^b+\epsilon)^{\frac{1}{b}}-a>0,(a^b-\epsilon)^{\frac{1}{b}}-a<0$,存在$N_1$,当$n>N_1$时

$(a^b-\epsilon)^{\frac{1}{b}}-a<a_n-a<(\epsilon+a^b)^{\frac{1}{b}}-a$,得$(a^b-\epsilon)^{\frac{1}{b}}<a_n<(a^b+\epsilon)^{\frac{1}{b}}$

所以$a^b-\epsilon<a_n^b<a^b+\epsilon$,即$|a_n^b-a^b|<\epsilon$

或者$a_n^b=e^{b\ln{a_n}}$,由(2),(3)结果知结论成立。

(5).

对任意的$0<\epsilon<\frac{\pi}{2}$,存在N,当n>N时,$|a_n-a|<\epsilon$,则:

$|\sin{a_n}-\sin{a}|=|2\cos\frac{a_n+a}{2}\sin\frac{a_n-a}{2}|\le2|\sin\frac{a_n-a}{2}|<2.|\frac{a_n-a}{2}|=\epsilon(0<x<\frac{\pi}{2},\sin{x}<x)$


5.设a>0,证明$\lim\limits_{n\to\infty}{\frac{\log_an}{n}}=0$

证:$\frac{\log_a{n}}{n}=\frac{1}{n}\log_a{n}=\log_a{\sqrt[n]{n}}$

根据上一题的结论,$\lim\limits_{n\to\infty}{\frac{\log_an}{n}}=\log_a{1}=0.(\lim\limits_{n\to\infty}\sqrt[n]{n}=1)$.

\end{document}